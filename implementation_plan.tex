\documentclass[12pt]{article}
\usepackage{amssymb}
\usepackage{amsthm}
\usepackage{dsfont}
\usepackage{amsmath}
\usepackage[latin1]{inputenc}
\usepackage[english]{babel}
\usepackage{graphicx}
\usepackage{bbm}
\usepackage{tikz}
\usepackage{color}  
%\usepackage{nath}
%\delimgrowth=1

\usepackage[margin=1in,footskip=0.25in]{geometry}
\linespread{1.3}

\usepackage{verbatim}
\usepackage{enumitem}

\newtheorem{theorem}{Theorem}%[section]
\newtheorem{lemma}[theorem]{Lemma}
\newtheorem{proposition}[theorem]{Proposition}
\newtheorem{corollary}[theorem]{Corollary}
\newtheorem{question}{Open question}
\newtheorem{defi}[theorem]{Definition}
\theoremstyle{plain}
\newtheorem{assumption}{Assumption}

\theoremstyle{remark}
\newtheorem{remark}{Remark}
\newtheorem{example}{\it Example\/}

\theoremstyle{definition}
\newtheorem{exercise}{Exercise}
\newtheorem{problem}{Problem}

\def\RRn{\mathbb{R}^n}
\def\RR{\mathbb{R}}
\def\RRN{\mathbb{R}^N}
\def\ZZ{\mathbb{Z}}
\def\ZZN{\mathbb{Z}^N}
\def\RRnm{\mathbb{R}^{n \times m}}
\def\RRmn{\mathbb{R}^{m \times n}}

\def\dd{\, \mathrm{d}}

\def\grad{\nabla}
\def\weakConv{\rightharpoonup}
\def\LL{\mathrm{L}}
\def\WW{\mathrm{W}}
\def\naturals{\mathbb{N}}

\def\intRRn{\int_{\RRn}}

\def\calM{\mathcal{M}}
\def\calE{\mathcal{E}_p}
\def\calC{\mathcal{C}}

\def\dist{\text{dist}}
\def\supp{\text{supp}}
\def\indyk{\mathds{1}}

\def\intOmega{\int_{\Omega}}
\def\intRRnm{\int_{\RRnm}}

\def\PP{\mathrm{P}}

\def\Ccinfty{C_c^{\infty}}

\DeclareMathOperator{\im}{Im}
\DeclareMathOperator{\id}{Id}
\renewcommand{\phi}{\varphi}
\renewcommand{\epsilon}{\varepsilon}
\renewcommand{\geq}{\geqslant}
\renewcommand{\leq}{\leqslant}
\renewcommand{\div}{\text{div}}

\newcommand{\measurerestr}{%
  \,\raisebox{-.127ex}{\reflectbox{\rotatebox[origin=br]{-90}{$\lnot$}}}\,%
}

\def\Xint#1{\mathchoice
{\XXint\displaystyle\textstyle{#1}}%
{\XXint\textstyle\scriptstyle{#1}}%
{\XXint\scriptstyle\scriptscriptstyle{#1}}%
{\XXint\scriptscriptstyle\scriptscriptstyle{#1}}%
\!\int}
\def\XXint#1#2#3{{\setbox0=\hbox{$#1{#2#3}{\int}$ }
\vcenter{\hbox{$#2#3$ }}\kern-.6\wd0}}
\def\ddashint{\Xint=}
\def\dashint{\Xint-}

\renewcommand{\phi}{\varphi}
\renewcommand{\epsilon}{\varepsilon}
\renewcommand{\geq}{\geqslant}
\renewcommand{\leq}{\leqslant}

\def\RRn{\mathbb{R}^n}
\def\RR{\mathbb{R}}
\def\RRN{\mathbb{R}^N}
\def\ZZ{\mathbb{Z}}
\def\ZZN{\mathbb{Z}^N}
\def\multiindices{\mathbb{Z}^N_+}
\def\RRnm{\mathbb{R}^{n \times m}}
\def\RRmn{\mathbb{R}^{m \times n}}
\def\naturals{\mathbb{N}}

\def\Ccinfty{C_c^{\infty}}

\def\dd{\, \mathrm{d}}
\def\grad{\nabla}
\def\weakConv{\rightharpoonup}

\def\intRRn{\int_{\RRn}}
\def\intOmega{\int_{\Omega}}
\def\intRRnm{\int_{\RRnm}}
\def\intRRN{\int_{\RRN}}

\def\RRNn{\RR^{N \times n}}

\def\CC{\mathrm{C}}

\def\RRn{\RR^n}
\def\RRm{\RR^m}

\usetikzlibrary{shapes, arrows, calc, arrows.meta, fit, positioning}  
\tikzset{  
    -Latex,auto,node distance =1.5 cm and 1.3 cm, thick,
    state/.style ={ellipse, draw, minimum width = 0.9 cm}, 
    point/.style = {circle, draw, inner sep=0.18cm, fill, node contents={}},  
    bidirected/.style={Latex-Latex}, 
    el/.style = {inner sep=2.5pt, align=right, sloped}  
}  

\def\hint{\noindent \textbf{Hint: }}
\begin{document}

\begin{center}
\begin{Large}
\textbf{LN fee bounds: Implementation plan}

\end{Large}
\text{sketch}
\end{center}


\subsubsection{parameters/class}
\begin{itemize}
    \item payments: $\lambda, c,p\in \RR$, sender address, reciever address, (add route later)
    \item Channels (parent class): $m, ibd, obd\in \RR$ (size and balances), payments, name
    \item Unidirectional channels: Channels, cost ($\sqrt{\frac{2B\lambda c}{r}}$)
    \item Bidirectional channels: Channels x 2, cost ($3(\frac{2B\lambda}{r})^{1/3}$)
    \item Node: $R\in \RR$ (revenue), address, list of channels, payments, 
    \item Network: nodes, $B, r\in \RR$, time
\end{itemize}


\subsubsection{overview}
The network contains nodes, each has a set of existing channels and payments. 

The simulation selects a node to make additional payment to a nonneighbor. Change $p$ and observe if the node should make a directed channel or use an intermediate node. For the intermediate node, observe how its revenue changes. 

\subsubsection{assumptions}
\begin{itemize}
    \item Intermediate node is responsible for the change in cost of the channels 
    \item The sender node pays all the transaction fee
    \item B, r are costant throughout channel lifetime
    \item channel lifetime and time between each transaction are random variables
\end{itemize}

\subsubsection{functions} 

channel: getNumTX ($k$), getLifetime ($\tau(\alpha)$), getTxFee ($T(\alpha)$), getOppoCost ($I(\alpha)$), getChannelCost (depends on the type of the channel)

\subsubsection{implementation}
At the start, build the network with 3 nodes: Alice, Bob, and Charlie. Construct an unidirectional channel between Alice and Bob, and then construct a bidirectional channel between Bob and Charlie. Set the average channel lifetime and frequency parameters. Also provide $B,r$. 
\\ At each round, generate the time between the transactions Alice makes. 
\\ Bob and Charlie's transaction depends on $\lambda$.
\\ The independent variable is p, the dependent varaible for Alice is the cost of direct channels and transactions fee, and the independent variable for Bob is his revenue. 
\\ Compare the result with lower and upper bound calculations
\end{document}

